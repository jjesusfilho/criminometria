\documentclass[12pt,]{article}
\usepackage[margin=1in]{geometry}
\newcommand*{\authorfont}{\fontfamily{phv}\selectfont}
\usepackage[]{mathpazo}
\usepackage{abstract}
\renewcommand{\abstractname}{}    % clear the title
\renewcommand{\absnamepos}{empty} % originally center
\newcommand{\blankline}{\quad\pagebreak[2]}

\providecommand{\tightlist}{%
  \setlength{\itemsep}{0pt}\setlength{\parskip}{0pt}} 
\usepackage{longtable,booktabs}

\usepackage{parskip}
\usepackage{titlesec}
\titlespacing\section{0pt}{12pt plus 4pt minus 2pt}{6pt plus 2pt minus 2pt}
\titlespacing\subsection{0pt}{12pt plus 4pt minus 2pt}{6pt plus 2pt minus 2pt}

\titleformat*{\subsubsection}{\normalsize\itshape}

\usepackage{titling}
\setlength{\droptitle}{-.25cm}

%\setlength{\parindent}{0pt}
%\setlength{\parskip}{6pt plus 2pt minus 1pt}
%\setlength{\emergencystretch}{3em}  % prevent overfull lines 

\usepackage[T1]{fontenc}
\usepackage[utf8]{inputenc}

\usepackage{fancyhdr}
\pagestyle{fancy}
\usepackage{lastpage}
\renewcommand{\headrulewidth}{0.3pt}
\renewcommand{\footrulewidth}{0.0pt} 
\lhead{}
\chead{}
\rhead{\footnotesize Análise quantitativa de processos criminais -- Janeiro de 2018}
\lfoot{}
\cfoot{\small \thepage/\pageref*{LastPage}}
\rfoot{}

\fancypagestyle{firststyle}
{
\renewcommand{\headrulewidth}{0pt}%
   \fancyhf{}
   \fancyfoot[C]{\small \thepage/\pageref*{LastPage}}
}

%\def\labelitemi{--}
%\usepackage{enumitem}
%\setitemize[0]{leftmargin=25pt}
%\setenumerate[0]{leftmargin=25pt}




\makeatletter
\@ifpackageloaded{hyperref}{}{%
\ifxetex
  \usepackage[setpagesize=false, % page size defined by xetex
              unicode=false, % unicode breaks when used with xetex
              xetex]{hyperref}
\else
  \usepackage[unicode=true]{hyperref}
\fi
}
\@ifpackageloaded{color}{
    \PassOptionsToPackage{usenames,dvipsnames}{color}
}{%
    \usepackage[usenames,dvipsnames]{color}
}
\makeatother
\hypersetup{breaklinks=true,
            bookmarks=true,
            pdfauthor={ ()},
             pdfkeywords = {},  
            pdftitle={Análise quantitativa de processos criminais},
            colorlinks=true,
            citecolor=blue,
            urlcolor=blue,
            linkcolor=magenta,
            pdfborder={0 0 0}}
\urlstyle{same}  % don't use monospace font for urls


\setcounter{secnumdepth}{0}





\usepackage{setspace}

\title{Análise quantitativa de processos criminais}
\author{José de Jesus Filho}
\date{Janeiro de 2018}


\begin{document}  

		\maketitle
		
	
		\thispagestyle{firststyle}

%	\thispagestyle{empty}


	\noindent \begin{tabular*}{\textwidth}{ @{\extracolsep{\fill}} lr @{\extracolsep{\fill}}}


E-mail: \texttt{\href{mailto:jjesusfilho@gmail.com}{\nolinkurl{jjesusfilho@gmail.com}}} & Web: \href{http://josejesus.info}{\tt josejesus.info}\\
Telefone: (11)98522-0210  &  Endereço: Rua Waldemar Adelino da Silva, 30\\
CNPJ: 23.268.449/0001-30 &  São Paulo - SP CEP 02929-020\\
	
	\hline
	\end{tabular*}
	
\vspace{2mm}
	


\section{Ementa}\label{ementa}

O curso irá introduzir a profissionais e estudantes de direito uma
metodologia para obter e analisar dados dos processos judiciais
criminais dos tribunais estaduais e federais. Inicialmente, serão
introduzidos aos fundamentos teóricos do realismo jurídico. os
participantes irão familiarizar-se com a linguagem de programação R e,
num segundo momento, irão apender técnicas de raspagem de dados na web
(webscraping), processamento de linguagem natural (natural languange
processing-NLP), limpeza e organização de base de dados. Na última
etapa, os participantes aprenderão as técnicas de análise descritiva e
preditiva de dados jurídicos.

O curso será eminentemente prático, as bases teóricas somente serão
fornecidas na medida em que sirvam para elucidar conceitos
imprecindíveis para a compreensão das operações realizadas.

\section{Programa do curso}\label{programa-do-curso}

\subsection{Aspectos teóricos}\label{aspectos-teoricos}

\begin{enumerate}
\def\labelenumi{\arabic{enumi}.}
\tightlist
\item
  Paradigmas de interpretação do comportamento judicial
\item
  Diferenças e complementariedade na pesquisa quanti e quali no direito
\item
  Introdução à jurimetria
\item
  Introdução à criminometria
\end{enumerate}

\subsection{Introdução ao R}\label{introducao-ao-r}

\begin{enumerate}
\def\labelenumi{\arabic{enumi}.}
\tightlist
\item
  Objetos no R: números e vetores
\item
  Outros objetos no R: fatores no R
\item
  Outros objetos no R: caracteres
\item
  Outros objetos no R: matrizes
\item
  Outros objetos no R: listas e data frames
\item
  Importando dados para o R
\item
  Exportando dados do R
\item
  Execução condicionada: ifelse
\item
  Execução condicionada: loops: for, repeat, while
\item
  Noções de distribuições probabilísticas
\item
  Estatística básica: média, mediana, moda, variância, desvio padrão,
  erro padrão
\item
  Criando funções no R
\end{enumerate}

\subsection{Coletando dados dos tribunais de
justiça}\label{coletando-dados-dos-tribunais-de-justica}

\begin{enumerate}
\def\labelenumi{\arabic{enumi}.}
\tightlist
\item
  Noções de html, xml, json e css
\item
  Procotolo http
\item
  Noção de ajax
\item
  Requisitando dados da web: httr e rvest
\item
  Extraindo conteúdo da página: rvest, xml2, css e xpath
\item
  Extraindo conteúdos dinâmicos: RSelenium
\item
  Baixando arquivos da web: pdf, csv, excel
\end{enumerate}

\subsection{Manuseio de textos}\label{manuseio-de-textos}

\begin{enumerate}
\def\labelenumi{\arabic{enumi}.}
\tightlist
\item
  Introdução ao processamento de linguagem natural
\item
  Conversão e importação dos textos
\item
  Expressões regulares
\item
  Extraindo informações relevantes das peças processuais
\item
  Tokenização, lemanenização, kwic e outras técnicas de manuseio de
  textos
\item
  Estruturando textos em tabelas, planilhas e matrizes
\end{enumerate}

\subsection{Manuseio da base de dados}\label{manuseio-da-base-de-dados}

\begin{enumerate}
\def\labelenumi{\arabic{enumi}.}
\tightlist
\item
  Introdução ao tidyverse
\item
  Manuseio de dados com dplyr
\item
  Programação funcional no R com purrr
\item
  Manuseio de datas: lubridate
\item
  Manuseio de fatores: forcats
\item
  Deixando a base pronta para análise
\end{enumerate}

\subsection{Introdução ao Aprendizado de Máquina (Machine Learning) no
R}\label{introducao-ao-aprendizado-de-maquina-machine-learning-no-r}

\begin{enumerate}
\def\labelenumi{\arabic{enumi}.}
\tightlist
\item
  Noção de Machine Learning
\item
  Aprendizado supervisionado vs aprendizado não supervisionado
\item
  Overfitting vs underfitting
\item
  Regularização
\item
  Noção de entropia
\item
  Regressão logística
\item
  Árvores de decisão
\item
  Floresta aleatória
\item
  Boosting: Adaboost, Gradient boosting, Extreme Boosting
\item
  Outros algorítimos: SVM, naive bayes, knn
\end{enumerate}

\subsection{Rodando modelos de predição das decisões
judiciais}\label{rodando-modelos-de-predicao-das-decisoes-judiciais}

\begin{enumerate}
\def\labelenumi{\arabic{enumi}.}
\tightlist
\item
  Dividir para conquistar: criando as bases de treinamento e teste
\item
  Estabelecendo hiperparâmetros
\item
  Validação cruzada
\item
  Rodando o modelo com regressão logística
\item
  Rodando o modelo com Floresta aleatória
\item
  Rodando o modelo com boosting
\item
  Calibrando os modelos
\end{enumerate}

\subsection{Avaliando o desempenho dos modelos
rodados}\label{avaliando-o-desempenho-dos-modelos-rodados}

\begin{enumerate}
\def\labelenumi{\arabic{enumi}.}
\tightlist
\item
  Comparando os modelos
\item
  Verificando as métricas de desempenho de cada modelo: matriz de
  confusão, acurácia, senstividade, especificidade,
\item
  Gráfico de Kolomogorov Smirnov
\item
  Área sob a curva: AUC-ROC
\item
  Coeficiente de GINI
\end{enumerate}

\subsection{Realizando predições}\label{realizando-predicoes}

\begin{enumerate}
\def\labelenumi{\arabic{enumi}.}
\tightlist
\item
  Aplicando a modelo sobre a base teste
\item
  Predizendo valores individuais
\item
  Interpretando as probabilidades
\end{enumerate}

\subsection{Material utilizado no
curso}\label{material-utilizado-no-curso}

Além do software gratuito e de seu IDE RStudio, o palestrante oferecerá
material de estudo e de exercícios.

\section{Duração do curso}\label{duracao-do-curso}

O curso durará no mínimo 40 horas, podendo chegar a 45 horas, a depender
do andamento na cobertura do programa.

\section{Custos}\label{custos}

Definir




\end{document}

\makeatletter
\def\@maketitle{%
  \newpage
%  \null
%  \vskip 2em%
%  \begin{center}%
  \let \footnote \thanks
    {\fontsize{18}{20}\selectfont\raggedright  \setlength{\parindent}{0pt} \@title \par}%
}
%\fi
\makeatother
